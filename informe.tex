\documentclass{udpreport}
\title{Topología Física y Lógica de una red LAN}
\author{Integrantes: Thomas Muñoz, Ignacio Yanjari, Dagoberto Navarrete, Ignacio López.}
\date{Marzo de 2016}
\usepackage{graphicx}
\graphicspath{ {img/} }
\udpschool{Escuela de Informática y Telecomunicaciones}

\begin{document}
\maketitle
\tableofcontents % Despliega el índice
\chapter{Introducción}
En este laboratorio se buscó reconocer la composición de la red del laboratorio de Informática, identificando el hardware de red y los elementos que forman parte de ella, ya sean computadores, routers, switch, cables, etc. Se realizó también un diagrama de red, el cual especifica la información de los dispositivos que forman parte de esta (IP y MAC de cada uno)"agregar"(lo que escriba yo no lo tomen en cuenta si sienten que esta mal! :c  Se realizó también un diagrama de red, el cual especifica la información tanto de los pc´s(IP y MAC de cada uno ), la informacion de el switch y path.)
\chapter{Actividades}
\section{Identificación de elementos de red}
"otra vez"(Para poder identificar el IP y MAC de cada computador, fue necesario abrir la terminal de cada Pc.(intentare ir al lab para sacar una imagen del terminal me la mandare a mi correo y luego le marcare la parte en donde esta la ip y el MAc))
El laboratorio cuenta con 18 computadores HP modelo EliteDesk 800 G1, conectados con cables UTP categoría 5e a un a un switch Cisco Catalyst 2960 el cual a su vez está conectado a un patch panel"revisar"(para luego ser conectador ordenadamente en el switch y ver un mejor orden al momento de ver alguna falla en algun dispositivo).
\section{Información de los dispositivos}
"{\bf-Tipo:} Computador de escritorio\\
{\bf-Modelo:}  HP EliteDesk 800 G1 con factor de forma reducido (ENERGY STAR)
(K6P73LT)\\
{\bf-Especificaciones:} Procesador:
Intel® Core™ i7-4790 con gráficos Intel HD 4600 (3,6 GHz, 8 MB de caché, 4 núcleos)\\
Memoria, estándar:
SDRAM DDR3 de 8 GB y 1600 MHz (1 x 8 GB)\\
Unidad interna:
SATA de 1 TB y 7200 rpm\\
Unidad óptica:
Grabadora SATA de DVD SuperMulti delgada\\
Gráficos:
Gráficos Intel HD 4600\\
Interfaz de red:
Conexión de red Intel I217LM GbE integrada\\
{\bf-Tipo:} Switch\\
{\bf-Modelo:} Cisco Catalyst 2960-24TT-L Switch\\
{\bf-Especificaciones:}Compatibilidad:
Modulo convertidor TwinGig\\
Fecha de salida:
18 de Septiembre de 2005\\
Dimensiones:
4.4 x 44.5 x 23.6 cm\\
Paquetes por segundo(Mpps):
6.6\\
Watt Power Consumption:	
75\\
AC/DC Support:
AC only\\
{\bf-Tipo:}Cable de conexion a red\\
{\bf-Modelo:}Fastlink 5e\\
{\bf-Especificaciones:}Cubierta y pares sin apantallar.\\
Excede los requerimientos propuestos por la normativa TIA /EIA 568 B .2 ,ISO/IEC 11801 Categoría 5E.\\
Retardante a la llama y cero halógenos según el Standard IEC 60332-3 Cat C.\\
Soporta aplicaciones de hasta 125 MHz de ancho de banda.\\
Codificación de colores para cada uno de los pares\\
Distribuido en cajas de 305 m con bobina interna para facilitar el tendido del cable.\\
Cumple con las normativas de medioambiente CE y RoHS.\\
{\bf-Tipo:} Patch Panel\\
{\bf-Modelo:} Siemon HD5-24 Cat 5e 24 puertos\\
{\bf-Especificaciones:}Estándares de red: IEEE 802.3, IEEE 802.3ab, IEEE 802.3u\\
Tecnología de cableado: 10/100/1000Base-T(X)\\
Características de red: LAN\\
Color del producto: Negro\\
Materiales: Metal\\
Montaje en rack: 1U\\"
%Aca deberia ir el modelo de el path y el swtich tambien el modelo de los cables conectados uno a uno con el switch
\section{Diagrama de red}
\includegraphics[width=\textwidth]{diag.png}
Como se observa en el diagrama de red los computadores estan conectados por el cable de red uno por uno 
a van conectados a el patch, el cual hara que los cables se ubiquen con una mejor distribucion fisica 
en el espacio.\\
Gracias a esta descripcion se plantea que la topologia utilizada en el laboratorio de informática
Este sistema es muy costoso siendo comparado junto con las topologias de anillo o tren, ya que para poder
conectar cada dispositivo a el switch se necesita un cable nuevo de el mismo tamaño,pero al mismo tiempo
aporta una caracteristica muy importante la cual es una facil deteccion de problemas y si es que algun
cable de red se llegara a romper o a tener problemas solo perdera la conexión  ese dispositivo.Aun así 
si es que el switch llegara a dañarse de alguna forma esto afectaria a el sistema completo ya que solo
depende de éste.

\chapter{Conclusión}
%Texto
\end{document}
