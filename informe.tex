\documentclass{udpreport}
\title{Topología Física y Lógica de una red LAN}
\author{Integrantes: Thomas Muñoz, Ignacio Yanjari, Dagoberto Navarrete, Ignacio López.}
\date{Marzo de 2016}
\usepackage{graphicx}
\graphicspath{ {img/} }
\udpschool{Escuela de Informática y Telecomunicaciones}

\begin{document}
\maketitle
\tableofcontents % Despliega el índice
\chapter{Introducción}
En este laboratorio se buscó reconocer la composición de la red del laboratorio de Informática, identificando el hardware de red y los elementos que forman parte de ella, ya sean computadores, routers, switch, cables, etc. Se realizó también un diagrama de red, el cual especifica la información de los dispositivos que forman parte de esta (IP y MAC de cada uno)"agregar"(lo que escriba yo no lo tomen en cuenta si sienten que esta mal! :c  Se realizó también un diagrama de red, el cual especifica la información tanto de los pc´s(IP y MAC de cada uno ), la informacion de el switch y path.)
\chapter{Actividades}
\section{Identificación de elementos de red}
"otra vez"(Para poder identificar el IP y MAC de cada computador, fue necesario abrir la terminal de cada Pc.(intentare ir al lab para sacar una imagen del terminal me la mandare a mi correo y luego le marcare la parte en donde esta la ip y el MAc))
El laboratorio cuenta con 18 computadores HP modelo EliteDesk 800 G1, conectados con cables UTP categoría 5e a un a un switch Cisco Catalyst 2960 el cual a su vez está conectado a un patch panel"revisar"(para luego ser conectador ordenadamente en el switch y ver un mejor orden al momento de ver alguna falla en algun dispositivo).
\section{Información de los dispositivos}
%Aca deberia ir el modelo de el path y el swtich tambien el modelo de los cables conectados uno a uno con el switch
\section{Diagrama de red}
\includegraphics[width=\textwidth]{diag.png}
%Texto
\chapter{Conclusión}
%Texto
\end{document}
